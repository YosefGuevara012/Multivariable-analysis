\documentclass{article}

\usepackage{Sweave}
\begin{document}
\Sconcordance{concordance:Cafe.tex:Cafe.Rnw:%
1 2 1 1 0 6 1 1 3 1 0 7 1 7 0 1 6 23 1 1 3 1 0 2 1 4 0 1 3 3 1 1 2 1 0 %
3 1 1 5 2 0 5 1 5 0 1 4 2 1 1 3 1 0 1 1 18 0 1 3 4 1 1 3 1 0 1 4 2 0 1 %
1 13 0 1 4 4 1 1 4 3 0 1 1 9 0 1 3 2 1 1 3 1 0 6 1 6 0 1 5 23 1 1 2 1 0 %
1 1 1 2 15 0 1 2 2 1 1 3 1 0 1 1 13 0 4 1 8 0 1 1 5 0 1 1 12 0 1 3 2 1 %
1 2 1 0 1 1 10 0 1 3 2 1 1 3 1 0 1 1 15 0 4 1 8 0 2 1 1 4 3 0 1 1 20 0 %
1 4 26 1 1 2 1 0 1 1 7 0 2 1 7 0 2 1 7 0 2 1 7 0 1 2 1 0 4 1 3 0 1 2 3 %
1 1 3 2 0 1 1 8 0 2 1 4 0 1 3 2 1 1 3 1 0 1 1 1 2 5 1 1 2 1 5 3 0 1 3 2 %
0 1 3 7 0 1 1 6 0 2 1 6 0 1 5 5 1}


\section{Visualización de Datos}
Código cargar los datos y construir la gráfica 3D con los cafés e imprimir la
matriz Y 

\begin{Schunk}
\begin{Sinput}
> library ( FactoClass ); 
> data ( cafe ); 
> Y <- cafe [1:10 ,1:3];
> par ( las =1); # grafica ;
> Y3D <- scatterplot3d (Y, main ="Y",type ="h",color =" black ",box=FALSE ,las =1);
> Y3D $ points3d (Y, pch =1);
> addgrids3d (Y, grid = c("xy","xz","yz"));
> cord2d <-Y3D$xyz.convert (Y) # convertir cordenadas 3D a 2D;
> # poner etiquetas ;
> # text ( cord2d , labels = rownames (Y), cex =0.8 , col=" black ",pos =3);
> # xtable (Y, digits =c(0 ,0 ,1 ,0)) # para tabular de LaTeX ;
> 
\end{Sinput}
\end{Schunk}

\begin{table}[ht]
\centering
\begin{tabular}{rrrr}
\hline
& Color & DA & EA \\
\hline
ExCl & 298 & 385.1 & 25 \\
C40M & 361 & 481.3 & 41 \\
C40C & 321 & 422.6 & 40 \\
C20M & 335 & 444.3 & 33 \\
C20C & 314 & 368.7 & 32 \\
ExOs & 186 & 346.6 & 28 \\
O40M & 278 & 422.6 & 43 \\
O40C & 238 & 403.0 & 42 \\
O20M & 226 & 368.7 & 36 \\
O20C & 210 & 368.7 & 35 \\
\hline
\end{tabular}
\end{table}

\section{Calculo del Centro de Gravedad}
Código para calcular y adicionar el centro de gravedad a la figura

\begin{Schunk}
\begin{Sinput}
> g <- colMeans (Y) # centro de gravedad ;
> Y3D$points3d (t(g), pch =19 , col=" darkgreen ",type = "h");
> text ( Y3D$xyz.convert (t(g)), labels ="g",pos =3, col=" black ", cex =1.3);
> 
\end{Sinput}
\end{Schunk}

\section{Centro de datos}
Centrar los datos, graficarlos en 3D e imprimir la matriz Yc

\begin{Schunk}
\begin{Sinput}
> n <- 10
> par ( las =1);
> unos <-rep (1,n); # vector de n unos
> Yc <- Y - unos%*%t(g); 
> # grafica de datos centrados
> 
> Yc3D <- scatterplot3d (Yc , main ="Yc",type ="h",color =" black ", box =FALSE , las =1);
> Yc3D$points3d (Yc , pch =1);
> addgrids3d (Yc , grid =c("xy", "xz", "yz" ));
> text ( Yc3D$xyz.convert (Yc), labels = rownames (Yc),cex =0.8 , col =" black ",pos =3);
> Yc3D $ points3d (t(c(0 ,0 ,0)) , pch =19 , col =" black ",type = "h");
> text ( Yc3D$xyz.convert (t(c(0 ,0 ,0))) , labels ="0",pos =3, col =" black ",cex =1.3);
> 
> 
\end{Sinput}
\end{Schunk}

\section{Distancia al centro de graverdad}

\begin{Schunk}
\begin{Sinput}
> Y_dist <- round(as.dist(dist(Y)),0);
> Y_dist
\end{Sinput}
\begin{Soutput}
     ExCl C40M C40C C20M C20C ExOs O40M O40C O20M
C40M  116                                        
C40C   46   71                                   
C20M   70   46   27                              
C20C   24  122   55   78                         
ExOs  118  221  155  178  130                    
O40M   46  102   43   62   66  120               
O40C   65  146   85  106   84   78   45          
O20M   75  176  109  133   88   46   75   37     
O20C   90  188  123  146  104   33   87   45   16
\end{Soutput}
\begin{Sinput}
> 
\end{Sinput}
\end{Schunk}

\section{Calculo de matriz de varianzas}

Multuplica la matriz de

\begin{Schunk}
\begin{Sinput}
> par ( las =1) # etiquetas de los dos ejes sean horizontales ;
> # Calculo de la matriz de Varianzas y covarianzas
> 
> V <-t(Yc) %*% as.matrix (Yc)/n;
> V
\end{Sinput}
\begin{Soutput}
         Color       DA     EA
Color 3105.810 1738.388  60.95
DA    1738.388 1560.188 129.36
EA      60.950  129.360  33.45
\end{Soutput}
\begin{Sinput}
> 
> # V = var (Y)*(n -1)/n; # Se usa porque R toma los datos como una muestra y no una población
\end{Sinput}
\end{Schunk}

\section{Calculo de la matriz diagonal}

Toma lo valores de la diagonal de la matriz de varianzas

\begin{Schunk}
\begin{Sinput}
> # Diagonalización de la Matriz
> 
> Dsigma <- diag ( sqrt ( diag (V ))); 
> round (diag(Dsigma ) ,1);
\end{Sinput}
\begin{Soutput}
[1] 55.7 39.5  5.8
\end{Soutput}
\begin{Sinput}
> 
\end{Sinput}
\end{Schunk}

\section{matriz normalizada X}

\begin{Schunk}
\begin{Sinput}
> X <-as.matrix (Yc) %*% solve (Dsigma); 
> colnames (X) <- colnames (Y);
> X3D <- scatterplot3d (X, main ="X",type ="h",box= FALSE );
> X3D$points3d (Yc , pch =1);
> addgrids3d (X, grid =c("xy","xz","yz" ));
> text ( X3D $ xyz.convert (X), labels = rownames (X),cex =0.8 , pos =3);
> X3D $ points3d (t(c(0 ,0 ,0)) , pch =19 , col=" black ",type ="h");
> # text ( X3D $ xyz.convert (t(c(0 ,0 ,0))) , labels ="0",pos =3, col=" black ",cex =0.8);
> # xtable (X, digits = rep (1 ,4)) # tabla para LaTeX ;
> 
\end{Sinput}
\end{Schunk}

\begin{table}[ht]
\centering
\begin{tabular}{rrrr}
\hline
& Color & DA & EA \\
\hline
ExCl & 0.4 & -0.4 & -1.8 \\
C40M & 1.5 & 2.0 & 1.0 \\
C40C & 0.8 & 0.5 & 0.8 \\
C20M & 1.0 & 1.1 & -0.4 \\
C20C & 0.7 & -0.8 & -0.6 \\
ExOs & -1.6 & -1.4 & -1.3 \\
O40M & 0.0 & 0.5 & 1.3 \\
O40C & -0.7 & 0.0 & 1.1 \\
O20M & -0.9 & -0.8 & 0.1 \\
O20C & -1.2 & -0.8 & -0.1 \\
\hline
\end{tabular}
\end{table}

\section{Distancias de los datos estandarizadso}
Distancias entre cafés cuando los datos están estandarizados(centrados y reducidos)

\begin{Schunk}
\begin{Sinput}
> X<-scale(Y); 
> X_dist <- round(as.dist(dist(X)),1);
> X_dist
\end{Sinput}
\begin{Soutput}
     ExCl C40M C40C C20M C20C ExOs O40M O40C O20M
C40M  3.7                                        
C40C  2.6  1.6                                   
C20M  2.0  1.6  1.3                              
C20C  1.2  3.2  1.8  1.9                         
ExOs  2.2  4.9  3.5  3.6  2.3                    
O40M  3.1  2.0  0.9  2.0  2.3  3.4               
O40C  3.0  2.8  1.5  2.4  2.2  2.8  0.8          
O20M  2.2  3.6  2.2  2.6  1.6  1.6  1.9  1.3     
O20C  2.3  3.9  2.4  2.8  1.8  1.3  2.2  1.5  0.3
\end{Soutput}
\end{Schunk}

\section{Obtencion de los valores y vectores propios}

\begin{Schunk}
\begin{Sinput}
> des <-eigen (V); 
> des # calculo de valores y vectores propios
\end{Sinput}
\begin{Soutput}
eigen() decomposition
$values
[1] 4238.94326  444.23103   16.27412

$vectors
           [,1]       [,2]        [,3]
[1,] 0.83785566  0.5409101  0.07358114
[2,] 0.54512571 -0.8219057 -0.16525428
[3,] 0.02891094 -0.1785702  0.98350233
\end{Soutput}
\begin{Sinput}
> lambda <-des$values
> U <- des$vectors # matriz con vectores propios en columnas
> rownames(U) <- rownames (V)
> colnames(U) <- c(" Eje1 "," Eje2 "," Eje3 "); round (U ,3)
\end{Sinput}
\begin{Soutput}
       Eje1   Eje2   Eje3 
Color  0.838  0.541  0.074
DA     0.545 -0.822 -0.165
EA     0.029 -0.179  0.984
\end{Soutput}
\begin{Sinput}
> lambda
\end{Sinput}
\begin{Soutput}
[1] 4238.94326  444.23103   16.27412
\end{Soutput}
\begin{Sinput}
> U
\end{Sinput}
\begin{Soutput}
           Eje1       Eje2        Eje3 
Color 0.83785566  0.5409101  0.07358114
DA    0.54512571 -0.8219057 -0.16525428
EA    0.02891094 -0.1785702  0.98350233
\end{Soutput}
\begin{Sinput}
> 
\end{Sinput}
\end{Schunk}

\section{}

\begin{Schunk}
\begin{Sinput}
> F <- X %*% U
> round ( sort (F[ ,1]) ,2)
\end{Sinput}
\begin{Soutput}
 ExOs  O20C  O20M  O40C  ExCl  C20C  O40M  C40C  C20M  C40M 
-2.04 -1.38 -1.15 -0.50  0.04  0.09  0.33  0.93  1.38  2.28 
\end{Soutput}
\begin{Sinput}
> 
\end{Sinput}
\end{Schunk}

\section{Código para obtener el primer plano factorial y la tabla}

\begin{Schunk}
\begin{Sinput}
> F <- X %*% U; 
> round (F ,2) # coordenadas sobre los nuevos ejes
\end{Sinput}
\begin{Soutput}
      Eje1   Eje2   Eje3 
ExCl   0.04   0.82  -1.60
C40M   2.28  -0.97   0.67
C40C   0.93  -0.15   0.70
C20M   1.38  -0.24  -0.50
C20C   0.09   1.09  -0.39
ExOs  -2.04   0.46  -1.11
O40M   0.33  -0.63   1.13
O40C  -0.50  -0.58   0.99
O20M  -1.15   0.16   0.15
O20C  -1.38   0.04  -0.04
\end{Soutput}
\begin{Sinput}
> plot (F[ ,1:2] , las =1, asp =1) # plano 12
> text (F[ ,1:2] , label = rownames (F), col=" black ",pos =2) # etiquetas
> abline (h=0,v=0, col =" darkgrey ") # ejes
> rowSums (F^2)-> d2;d2 # distancias
\end{Sinput}
\begin{Soutput}
    ExCl     C40M     C40C     C20M     C20C     ExOs     O40M     O40C 
3.246622 6.578011 1.378697 2.226649 1.340566 5.614494 1.779107 1.572724 
    O20M     O20C 
1.359404 1.903726 
\end{Soutput}
\begin{Sinput}
> 1/n*F^2 %*%diag (1/ lambda )*100 -> cont # contribuciones
> F^2/d2*100 -> cos2 # cosenos cuadrados
> # tabla de ayudas para la interpretacion
> Ayu <-cbind ( dis2 =d2 ,F1=F[ ,1] , F2=F[ ,2] , cont1 = cont [,1] , cont2 =
+ cont [,2] , cos21 = cos2 [ ,1] , cos22 = cos2 [,2] ,
+ cosp = rowSums ( cos2 [ ,1:2]))
> round (Ayu ,2) # ayudas en consola
\end{Sinput}
\begin{Soutput}
     dis2    F1    F2 cont1 cont2 cos21 cos22  cosp
ExCl 3.25  0.04  0.82  0.00  0.02  0.06 20.75 20.81
C40M 6.58  2.28 -0.97  0.01  0.02 78.87 14.21 93.08
C40C 1.38  0.93 -0.15  0.00  0.00 63.26  1.57 64.83
C20M 2.23  1.38 -0.24  0.00  0.00 86.08  2.62 88.70
C20C 1.34  0.09  1.09  0.00  0.03  0.61 88.10 88.71
ExOs 5.61 -2.04  0.46  0.01  0.00 74.38  3.79 78.17
O40M 1.78  0.33 -0.63  0.00  0.01  6.30 22.38 28.68
O40C 1.57 -0.50 -0.58  0.00  0.01 15.71 21.62 37.32
O20M 1.36 -1.15  0.16  0.00  0.00 96.57  1.87 98.43
O20C 1.90 -1.38  0.04  0.00  0.00 99.84  0.09 99.93
\end{Soutput}
\begin{Sinput}
> #xtable (Ayu , digits = rep (2 ,9)) # salida para LaTeX
> 
\end{Sinput}
\end{Schunk}

% latex table generated in R 3.6.1 by xtable 1.8-4 package
% Sun Sep 13 19:22:39 2020
\begin{table}[ht]
\centering
\begin{tabular}{rrrrrrrrr}
\hline
& dis2 & F1 & F2 & cont1 & cont2 & cos21 & cos22 & cosp \\
\hline
ExCl & 3.25 & 0.04 & 0.82 & 0.00 & 0.02 & 0.06 & 20.75 & 20.81 \\
C40M & 6.58 & 2.28 & -0.97 & 0.01 & 0.02 & 78.87 & 14.21 & 93.08 \\
C40C & 1.38 & 0.93 & -0.15 & 0.00 & 0.00 & 63.26 & 1.57 & 64.83 \\
C20M & 2.23 & 1.38 & -0.24 & 0.00 & 0.00 & 86.08 & 2.62 & 88.70 \\
C20C & 1.34 & 0.09 & 1.09 & 0.00 & 0.03 & 0.61 & 88.10 & 88.71 \\
ExOs & 5.61 & -2.04 & 0.46 & 0.01 & 0.00 & 74.38 & 3.79 & 78.17 \\
O40M & 1.78 & 0.33 & -0.63 & 0.00 & 0.01 & 6.30 & 22.38 & 28.68 \\
O40C & 1.57 & -0.50 & -0.58 & 0.00 & 0.01 & 15.71 & 21.62 & 37.32 \\
O20M & 1.36 & -1.15 & 0.16 & 0.00 & 0.00 & 96.57 & 1.87 & 98.43 \\
O20C & 1.90 & -1.38 & 0.04 & 0.00 & 0.00 & 99.84 & 0.09 & 99.93 \\
\hline
\end{tabular}
\end{table}


\section{Calcular las coordenadas factoriales de los dos cafés comerciales
y su proyección sobre el primer plano factorial}

\begin{Schunk}
\begin{Sinput}
> comer <-as.matrix ( cafe [11:12 ,1:3]);
> comer
\end{Sinput}
\begin{Soutput}
     Color    DA EA
Com1   221 413.3 27
Com2   264 400.9 23
\end{Soutput}
\begin{Sinput}
> comc <-comer - rep (1 ,2) %*%t(g); 
> comc # centrado
\end{Sinput}
\begin{Soutput}
     Color    DA    EA
Com1 -55.7 12.14  -8.5
Com2 -12.7 -0.26 -12.5
\end{Soutput}
\begin{Sinput}
> comcr <- comc %*%solve ( Dsigma ) # reducido
> colnames ( comcr ) <- colnames ( comer ); comcr
\end{Sinput}
\begin{Soutput}
          Color           DA        EA
Com1 -0.9994654  0.307347839 -1.469674
Com2 -0.2278853 -0.006582408 -2.161285
\end{Soutput}
\begin{Sinput}
> Fsup <- comcr %*%U; 
> Fsup
\end{Sinput}
\begin{Soutput}
          Eje1       Eje2      Eje3 
Com1 -0.7123542 -0.5307919 -1.569760
Com2 -0.2570080  0.2680857 -2.141309
\end{Soutput}
\begin{Sinput}
> # primer plano factorial
> plot (F[ ,1:2] , las =1, asp =1)
> text (F[ ,1:2] , label = rownames (F), col=" black ",pos =1)
> abline (h=0,v=0, col=" darkgrey ")
> points (Fsup ,col =" black ",pch =20) # cafes comerciales
> text (Fsup , labels =c(" Com1 "," Com2 "), col=" darkgreen ",pos =2)
\end{Sinput}
\end{Schunk}

\section{Código para calcular las coordenadas de las categorías de contaminación y
proyectarlas sobre el primer plano factorial}

\begin{Schunk}
\begin{Sinput}
> conta <-factor (c(" exce "," maiz "," ceba "," maiz "," ceba "," exce ",
+ " maiz "," ceba "," maiz "," ceba "))
> centroids (F, conta )$ centroids -> Fconta ; Fconta
\end{Sinput}
\begin{Soutput}
            Eje1        Eje2       Eje3 
 ceba  -0.2128594  0.09945118  0.3161817
 exce  -0.9999009  0.64113648 -1.3552319
 maiz   0.7128099 -0.42001942  0.3614343
\end{Soutput}
\begin{Sinput}
> points ( Fconta ,col =" brown ",pch =20)
> text ( Fconta , col=" brown ",labels = rownames ( Fconta ),pos =2)
> 
\end{Sinput}
\end{Schunk}

\section{}

\begin{Schunk}
\begin{Sinput}
> G <-U %*% diag(sqrt(lambda)); 
> G <- G / G # G <- cor (Y,F)
> colnames (G) <-c("G1","G2","G3");
> G3D <- scatterplot3d (G, main ="G")
> coord <- G3D$xyz.convert(G)
> text (coord , labels = rownames (G), cex =0.8 , col=" black ",pos =4)
> G3D$ plane (0 ,0 ,0 , col =" darkgrey ")
> G3D$ points3d (t(c(0 ,0 ,0)) , pch =19 , col=" black ")
> cero <- G3D$xyz.convert (0, 0, 0)
> for (eje in 1:3) {
+ arrows ( cero $x, cero $y, coord $x[ eje ], coord $y[eje ], lwd = 2,
+ length = 0.1)
+ }
> # dev . print ( device =xfig , file =" cafeEspera .fig ")# grafica en xfig
> 
> s.corcircle (G, clabel =2);
> # proyeccion de nota como variable ilustrativa
> 
> Nota <- cafe [1:10 ,16]; Nota ;
\end{Sinput}
\begin{Soutput}
 [1] 7.46 6.24 6.12 6.04 6.22 7.40 5.90 6.94 6.90 7.16
\end{Soutput}
\begin{Sinput}
> Fnota <- cor (Nota ,F); Fnota ;
\end{Sinput}
\begin{Soutput}
          Eje1      Eje2       Eje3 
[1,] -0.7328713 0.4366652 -0.5836551
\end{Soutput}
\begin{Sinput}
> arrows (0,0 , Fnota [1] , Fnota [2] , col =" black ",angle =10 , lty =2);
> text (Fnota ," Nota ",col =" black ",pos =1, cex =2, font =3);
> 
> 
> 
\end{Sinput}
\end{Schunk}





\end{document}
